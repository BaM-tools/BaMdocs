\documentclass[a4paper]{article}
\usepackage[utf8]{inputenc}
\usepackage{tcolorbox}
\usepackage{courier}
\usepackage[margin=1in]{geometry}
\usepackage{amsmath}
\usepackage{hyperref}

\title{Model: name, acronym}
\author{Author(s) of this sheet: name, surname}
\date{Last update: month year}

\begin{document}

\maketitle

\section{General description}
Describe in plain words the objectives of the model, its main input/output variables, its general principles.\\
Provide key scientific references and documentation (example: \cite{Doe2015}), links to online contents if any (example: \url{https://forge.irstea.fr/projects/bam}), etc.

\section{Key properties}

\begin{description}
\item[ID] \texttt{"ModelID"}, the ID of the model in BaM.

\item[Input Variables] $N_X=?$
\begin{enumerate}
\item Name of input variable 1 $[\rm unit]$: comments.
\item Name of input variable 2 $[\rm unit]$: comments.
\item etc.
\end{enumerate}

\item[Output Variables] $N_Y=?$
\begin{enumerate}
\item Name of output variable 1 $[\rm unit]$: comments.
\item Name of output variable 2 $[\rm unit]$: comments.
\item etc.
\end{enumerate}

\item[Parameters] $N_{par}=?$
\begin{enumerate}
\item Name of parameter 1 $[\rm unit]$: comments.
\item Name of parameter 2 $[\rm unit]$: comments.
\item etc.
\end{enumerate}

\item[Derived parameters] $N_{d}=?$
\begin{enumerate}
\item Name of derived parameter 1 $[\rm unit]$: comments.
\item Name of derived parameter 2 $[\rm unit]$: comments.
\item etc.
\end{enumerate}

\item[State variables] $N_{Z}=?$
\begin{enumerate}
\item Name of state variable 1 $[\rm unit]$: comments.
\item Name of state variable 2 $[\rm unit]$: comments.
\item etc.
\end{enumerate}

\item[Xtra information] Explain what kind of extra information is needed to run the model (if any).
\begin{enumerate}
\item Use a numbered list if there are several items.
\item etc.
\end{enumerate}

\end{description}

\section{Mathematical formulation}
Provide all formulas behind the model, using equations.
%
\begin{equation}
Y=f(X,\theta)
\label{eq:formula}
\end{equation}
%
If possible try to provide all needed information to make the model fully reproducible (i.e. re-implementable through other means, using the information contained in this sheet only).\\
If this is not feasible due to the complexity of the model, provide references \cite{Doe2015} or links to some online content (\url{https://forge.irstea.fr/projects/bam}).\\
Figures and tables are welcome too!
%
\begin{center}
   \begin{figure} [!htbp]
    %\includegraphics[width=\textwidth]{Figure1.pdf}
    \caption{Figure caption}
    \label{fig:figure}
   \end{figure}
 \end{center}
%
\begin{table}[!htbp]
	\centering
	\begin{tabular}{ccc}
		\hline
		C1 & C2 & C3 \\
		\hline
		blah & blah & blah \\
		\hline
		blah & blah & blah\\
		\hline
	\end{tabular}
	\caption{Table caption}
	\label{tab:table}
\end{table}
%

\section{Configuration in BaM}
The example below shows a typical configuration file for this model.\\
%
\begin{tcolorbox}
	\begin{tabular}{ll}
		\texttt{"ID"} & \texttt{! Model ID}\\
		\texttt{2} & \texttt{! nX: number of input variables}\\
		\texttt{1} & \texttt{! nY: number of ouput variables}\\
		\texttt{4} & \texttt{! nPar: number of parameters theta}\\
		\texttt{"Par1"} & \texttt{! Parameter name}\\
		\texttt{0.0} & \texttt{! Initial guess}\\
		\texttt{"Gaussian"} & \texttt{! Prior distribution}\\
		\texttt{0.0,1.0} & \texttt{! Prior parameters}\\
		\texttt{"Par2"} & \texttt{! Parameter name}\\
		\texttt{1.0} & \texttt{! Initial guess}\\
		\texttt{"LogNormal"} & \texttt{! Prior distribution}\\
		\texttt{0.0,0.2} & \texttt{! Prior parameters}\\
		\texttt{"Par3\_gravity"} & \texttt{! Parameter name}\\
		\texttt{9.81} & \texttt{! Initial guess}\\
		\texttt{"FIX"} & \texttt{! Prior distribution}\\
		\texttt{0} & \texttt{! Prior parameters}\\
		\texttt{"Par4\_variable"} & \texttt{! Parameter name}\\
		\texttt{-99.99} & \texttt{! Initial guess}\\
		\texttt{"VAR"} & \texttt{! Prior distribution}\\
		\texttt{"Config\_Par4\_VAR.txt"} & \texttt{! Prior parameters}\\
	\end{tabular}
\end{tcolorbox}
%
If an Xtra configuration file is also needed:\\
%
\begin{tcolorbox}
	\begin{tabular}{ll}
		\texttt{"blahblah"} & \texttt{! meaning of line 1 }\\
		\texttt{"blahblah"} & \texttt{! meaning of line 2 }\\
	\end{tabular}
\end{tcolorbox}

\begin{thebibliography}{00}

\bibitem[Doe et al. (2015)]{Doe2015}
Doe, J, 2015. Paper Title. Journal name. doi:10.5194/hess-19-2427-2015.

\end{thebibliography}

\end{document}
