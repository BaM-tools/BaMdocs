\documentclass[a4paper]{article}
\usepackage[utf8]{inputenc}
\usepackage{tcolorbox}
\usepackage{courier}
\usepackage[margin=1in]{geometry}
\usepackage{amsmath}
\usepackage{hyperref}

\title{Text File Model: define your own model as a simple text formula}
\author{Author(s) of this sheet: Benjamin Renard}
\date{Last update: March 2020}

\begin{document}

\maketitle

\section{General description}
This pseudo-model provides an interface to define any model that is simple enough to be written as a one-line text formula (or several one-line formulas if there are several output variables).

\section{Key properties}

\begin{description}
\item[ID] \texttt{"TextFile"}.

\item[Input Variables]  any number $N_X \geq0$ is allowed.

\item[Output Variables]  any number $N_Y \geq1$ is allowed.

\item[Parameters] any number $N_{par} \geq0$ is allowed.

\item[Derived parameters] $N_{d}=0$ - this utility is not available for this model.

\item[State variables] $N_{Z}=0$ - this utility is not available for this model.

\item[Xtra information] The model is defined using the Xtra-information configuration file (the one declared in the fourth row of Config\_BaM.txt). This file should contain the following lines: 
\begin{enumerate}
\item The number of input variables $N_X$; ex.: \texttt{3}.
\item The names of all input variables, comma-separated; ex.: \texttt{A, B, C}.
\item The number of parameters $N_{par}$; ex.: \texttt{4}.
\item The names of all parameters, comma-separated; ex.: \texttt{par1, par2, par3, par4}.
\item The number of output variables $N_Y$; ex.: \texttt{2}.
\item The formula to compute the first output; ex: \texttt{par1+par2*exp(-1*par3*A*B)}.
\item $\dots$
\item The formula to compute the $N_Y^{th}$ output; ex: \texttt{A\^{}2*C\^{}par4}.
\end{enumerate}
\end{description}

\section{Mathematical formulation}

The formulas defining the model should be valid mathematical expressions that combine the following elements:

\begin{enumerate}
\item Operators \texttt{+}, \texttt{-}, \texttt{*}, \texttt{/}, \texttt{\^{}}; for the latter power operator, \texttt{**} is also accepted.
\item Numbers (use the point as decimal separator).
\item Input and parameter names.
\item Brackets (only use round brackets: \texttt{()}).
\item Standard mathematical functions. Currently available functions are:  \texttt{abs}, \texttt{exp}, \texttt{log10}, \texttt{log}, \texttt{sqrt}, \texttt{sin}, \texttt{cos}, \texttt{tan}, \texttt{asin}, \texttt{acos}, \texttt{atan}, \texttt{sinh}, \texttt{cosh}, \texttt{tanh}.
\item indicator functions. Currently available functions are: \texttt{is0} ($is0(x)=1$ if $x=0$, $0$ otherwise), \texttt{ispos} ($ispos(x)=1$ if $x\geq0$, $0$ otherwise), \texttt{isneg} (same for negative),  \texttt{isspos} (same for strictly positive), \texttt{issneg} (same for strictly negative).
\end{enumerate}

The usual mathematical priority rules apply - when in doubt, use extra brackets. Please also consider the following tips to ensure formulas can be correctly interpreted and computed:

\begin{itemize}
\item Make sure there are no duplicates in input and parameter names.
\item Make sure all variables used in formulas have been declared as either input or parameter names.
\item Make sure all formulas are mathematically valid - in particular check brackets are balanced.
\item use indicator functions to avoid mathematically-impossible computations. For instance if some input or parameter \texttt{x} can take negative values, the formula \texttt{sqrt(x)} may be problematic; a possible way around this is to use the formula \texttt{sqrt(ispos(x)*x)}.
\item indicator functions can also be used to compute piecewise formulas. For instance:  \texttt{ispos(x)*x + issneg(x)*x\^{}2} returns $x$ if $x\geq0$ and $x^2$ if $x<0$.
\item Do not use tabulations; use spaces instead if you want to format the formula to make it easier to read or indent the configuration file.
\end{itemize}

Finally, if you need a mathematical function that is not available yet, feel free to send a request to \href{mailto:bam.dev@lists.irstea.fr}{bam.dev@lists.irstea.fr}. Note however that only simple mathematical functions of a single variable can be added. For instance:

\begin{itemize}
\item the function $max(x,y)$ cannot be added - it is a function of two variables.
\item the probability density function (pdf) of a standard normal distribution $\phi(x)$ can be added; however its parameterized version $\phi(x;\mu,\sigma)$ cannot.  
\end{itemize}

\section{Configuration in BaM}
The example below shows an example of an Xtra-information configuration file to specify a population dynamics model. The model assumes that the population of a given species at time $t$ is equal to:

%
\begin{equation}
P(t)=\frac{K}{1+\frac{K-p_0}{p_0}*exp(-r*T*t)}
\label{eq:formula}
\end{equation}
%
Where $p_0>0$ is the initial population, $K>p_0$ is the population upper limit (linked to environmental constraints), $T>0$ is the temperature time series (assumed to be always positive here) and $r$ is the growth-to-temperature ratio (i.e. the growth rate is equal to $r*T$ - the warmer the temperature, the larger the population growth). \\
Assume that the same initial population $p_0$ of the same species (hence having the same $r$) is placed in two different environments (hence different upper limits $K_1$ and $K_2$ and different temperatures times series $T_1$ and $T_2$). The 2-output model specified in the Xtra-information configuration file below allows computing the evolution of both populations through time:\\

\begin{tcolorbox}
	\begin{tabular}{ll}
		\texttt{3} & \texttt{! number of input variables}\\
		\texttt{t,T1,T2} & \texttt{! time, Temp. site 1, Temp. site 2}\\
		\texttt{4} & \texttt{! number of parameters}\\
		\texttt{P0,r,K1,K2} & \texttt{! P0, r, limit site 1, limit site 2}\\
		\texttt{2} & \texttt{! number of output variables}\\
		\texttt{K1/(1+((K1-P0)/P0)*exp(-r*T1*t))} & \texttt{! P(t) at site 1}\\
		\texttt{K2/(1+((K2-P0)/P0)*exp(-r*T2*t))} & \texttt{! P(t) at site 2}\\
	\end{tabular}
\end{tcolorbox}

All other configuration files work as for any other model - you just need to make sure that they are consistent with the specifications above (number of inputs/outputs/parameters, name and order of parameters). For instance the model configuration file could be as follows:

\begin{tcolorbox}
	\begin{tabular}{ll}
		\texttt{"TextFile"} & \texttt{! Model ID}\\
		\texttt{3} & \texttt{! nX: number of input variables}\\
		\texttt{2} & \texttt{! nY: number of ouput variables}\\
		\texttt{4} & \texttt{! nPar: number of parameters theta}\\
		\texttt{"P0"} & \texttt{! Parameter name}\\
		\texttt{100} & \texttt{! Initial guess}\\
		\texttt{"FIX"} & \texttt{! Prior distribution}\\
		\texttt{} & \texttt{! Prior parameters}\\
		\texttt{"r"} & \texttt{! Parameter name}\\
		\texttt{0.001} & \texttt{! Initial guess}\\
		\texttt{"LogNormal"} & \texttt{! Prior distribution}\\
		\texttt{-7.0,1.0} & \texttt{! Prior parameters}\\
		\texttt{"K1"} & \texttt{! Parameter name}\\
		\texttt{10000} & \texttt{! Initial guess}\\
		\texttt{"LogNormal"} & \texttt{! Prior distribution}\\
		\texttt{9.0,1.0} & \texttt{! Prior parameters}\\
		\texttt{"K2"} & \texttt{! Parameter name}\\
		\texttt{10000} & \texttt{! Initial guess}\\
		\texttt{"LogNormal"} & \texttt{! Prior distribution}\\
		\texttt{9.0,1.0} & \texttt{! Prior parameters}\\	\end{tabular}
\end{tcolorbox}

\end{document}
