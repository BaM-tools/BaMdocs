\documentclass[a4paper]{article}
\usepackage[utf8]{inputenc}
\usepackage{tcolorbox}
\usepackage{courier}
\usepackage[margin=1in]{geometry}
\usepackage{amsmath}
\usepackage{hyperref}

\title{Model: General Linear Model GLM}
\author{Author(s) of this sheet: Benjamin Renard}
\date{Last update: March 2019}

\begin{document}

\maketitle

\section{General description}
This model corresponds to the classical GLM approach used in statistics, a.k.a. multiple linear regression. It is a multivariate version of GLM (i.e. several output variables are allowed); however, the residuals will be considered as independent between each output variable (this is a current general restriction of BaM).\\
WARNING: in the GLM context, linearity is \emph{with respect to parameters} (as opposed to input variables). For instance, the regression $y=\theta_0+\theta_1x+\theta_2x^2+\theta_3 cos(x)$ is a linear regression between the 4 input variables $\left( 1,x,x^2,cos(x) \right)$ and $y$. However, the regression $y=cos( \theta x)$ is not linear.

\section{Key properties}

\begin{description}
\item[ID] \texttt{"Linear"}.

\item[Input Variables] any number $N_X \geq1$ is allowed.

\item[Output Variables] any number $N_Y \geq1$ is allowed

\item[Parameters] $N_{par}=N_X N_Y$

\item[Derived parameters] none, $N_{d}=0$

\item[State variables] none, $N_{Z}=0$

\item[Xtra information] none

\end{description}

\section{Mathematical formulation}
The GLM model can be written in the following matrix form:
%
\begin{equation}
\hat{\boldsymbol{y}}=\boldsymbol{x}\boldsymbol{\theta}
\label{eq:GLM}
\end{equation}
%
where $\boldsymbol{x}$ is the $N \times N_X$ matrix containing input variables (a.k.a. predictors or covariates), $\boldsymbol{\theta}$ is the parameter vector reorganized in a $N_X \times N_Y$ matrix, and $\hat{\boldsymbol{y}}$ is the $N \times N_Y$ matrix containing the predicted output variables (a.k.a. predictands).\\
As an illustration, the most common situation is the GLM model for predicting $N_Y=1$ output variable using several predictors: 
%
\begin{equation}
\hat{y}_i=\theta_0+\theta_1 x_{1,i}+\theta_2 x_{2,i}, \quad i=1, \dots ,N
\label{eq:example}
\end{equation}
%
This can be re-written in the matrix form of equation~(\ref{eq:GLM}) by posing:
%
\begin{equation}
\hat{\boldsymbol{y}}=
\begin{pmatrix}
   \hat{y}_1 \\ \vdots \\ \hat{y}_N 
\end{pmatrix}
,\quad
\boldsymbol{x}=
\begin{pmatrix}
  1 & x_{1,1} & x_{2,1} \\
  \vdots & \vdots & \vdots \\
  1 & x_{1,N} & x_{2,N}
\end{pmatrix}
,\quad
\boldsymbol{\theta}=
\begin{pmatrix}
  \theta_0 \\ \theta_1 \\ \theta_2 
\end{pmatrix}
\label{eq:example2}
\end{equation}

\section{Configuration in BaM}
The example below shows a configuration file for the model of equation~(\ref{eq:example}).\\
%
\begin{tcolorbox}
	\begin{tabular}{ll}
		\texttt{"Linear"} & \texttt{! Model ID}\\
		\texttt{3} & \texttt{! nX: number of input variables}\\
		\texttt{1} & \texttt{! nY: number of ouput variables}\\
		\texttt{3} & \texttt{! nPar: number of parameters theta}\\
		\texttt{"intercept"} & \texttt{! Parameter name}\\
		\texttt{0.0} & \texttt{! Initial guess}\\
		\texttt{"Gaussian"} & \texttt{! Prior distribution}\\
		\texttt{0.0,1.0} & \texttt{! Prior parameters}\\
		\texttt{"theta1"} & \texttt{! Parameter name}\\
		\texttt{0.0} & \texttt{! Initial guess}\\
		\texttt{"Gaussian"} & \texttt{! Prior distribution}\\
		\texttt{0.0,0.2} & \texttt{! Prior parameters}\\
		\texttt{"theta2"} & \texttt{! Parameter name}\\
		\texttt{0} & \texttt{! Initial guess}\\
		\texttt{"FlatPrior"} & \texttt{! Prior distribution}\\
		\texttt{} & \texttt{! Prior parameters}\\
	\end{tabular}
\end{tcolorbox}

\begin{thebibliography}{00}

\bibitem[Doe et al. (2015)]{Doe2015}
Doe, J, 2015. Paper Title. Journal name. doi:10.5194/hess-19-2427-2015.

\end{thebibliography}

\end{document}
